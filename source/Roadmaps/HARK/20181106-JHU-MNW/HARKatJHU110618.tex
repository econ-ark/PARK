% !TeX spellcheck = en_GB
       \documentclass[11pt]{beamer}
%	\usetheme{warsaw}
            \usepackage{setspace}
            \usepackage{graphicx} %draft option suppresses graphics dvi display
            \newcommand{\Prob}{\operatorname{Prob}}
            \clubpenalty 5000
            \widowpenalty 5000
            \renewcommand{\baselinestretch}{1.23}
            \usepackage{amsmath}
            \usepackage{amsthm}
            \usepackage{amsfonts}
            \usepackage{amssymb}
            \usepackage{bbm}
            \usepackage{cancel}
	 \newcommand{\E}{\mathbb{E}}
	 \newcommand{\R}{\mathbb{R}}
	 \newcommand{\pd}[2]{\frac{\partial#1}{\partial#2}}
	\newcommand{\bi}{\begin{itemize}}
	\newcommand{\ei}{\end{itemize}}
	\newcommand{\Die}{\mathsf{D}}
	\newcommand{\Live}{\cancel{\Die}}

\author{Matthew N. White, University of Delaware}

\title{Modeling Development Priorities for the \\ Heterogeneous Agents Resources and toolKit}

\date{November 6, 2018}

\begin{document}

% ========== Intro slide =================
\begin{frame}
\maketitle
\end{frame}


\begin{frame}
\frametitle{What This Talk is (Not) About}

\bi
\item <1->Lots of margins on which to improve HARK...

\item <1->...but you (should) care about your dissertation

\item <2->Going to focus on areas for expansion that ``integrate'' with a quality job market paper: new models

\item <3->NOT writing better documentation and tutorials

\item <4->NOT incorporating Numba or OpenCL (or speed generally)

\item <5->NOT structural changes or object representation

\item <5->UNLESS closely related to a methodological issue

\ei
\end{frame}



\begin{frame}
\frametitle{The Big Methodological Issues}

The same methodological/numeric issues come up in a wide array of models; details differ on case-by-case basis
\begin{enumerate}
\item <1->Curse of dimensionality: Size of problem increases exponentially in number of state (and control) variables

\item <2->Non-convex value function: FOCs necessary but not sufficient

\item <3->Discont policy function; can arise from (2) or discrete choice

\item <4->Kinked value function: how to integrate marginal value?

\item <5->Extrapolation errors compound over time

\item <6->Dimensionality ``mismatch'' between states and controls
\end{enumerate}

\end{frame}


\begin{frame}
\frametitle{The Big Issues: Dimensionality}

\bi
\item <1->$N$ state dimensions, 32 gridpoints on each $\Longrightarrow 32^N$ points!

\item <2->Solution: (adaptive) sparse grids.  Only add gridpoints where they ``matter'' or are ``informative'' relative to sparser grid

\item <3->Complication: How to interact with endogenous grid method?

\item <3->Maybe: Delaunay?  Delaunay with initial structure guess?

\item <4->But what about non-concave value functions?

\ei
\end{frame}


\begin{frame}
\frametitle{The Big Issues: Non-Concave Value Function}

\bi
\item <1->When value function is non-concave, naive EGM ``value function'' loops back on itself. Not a function!

\item <1->True value function is upper envelope of that

\item <2->1D, Fella: Find bounds of convex region, look for exact discontinuity point

\item <2->1D, Rust et al: Just use lots of points, implement discontinuities approximately

\item <3->2D, Jorgensen \& Druedahl: ``Reinterpolate'' EGM triangles on dense exogenous grid, use approximate discontinuities

\item <3->White: Got that to work in 3D as well

\item <4->Still no way to get exact location of discontinuity in 2D+... and ``approximate'' discont might not be that precise!

\ei
\end{frame}


\begin{frame}
\frametitle{The Big Issues: Integrating Discontinuous Functions}

\bi
\item <1->Numeric integration is summation of ``enough'' discrete points

\item <2->Theorems that bound approximation errors depend on continuity and smoothness (of all derivatives)...

\item <2->...But discontinuous functions fail that test hard

\item <3->Getting error bounds on these integrals requires finding discontinuity and separately integrating on either side

\item <4->But what about integrating in 2D+ space?

\item <4->Or with discontinuities that propagate?

\item <5->I got nothin'

\ei
\end{frame}


\begin{frame}
\frametitle{The Big Issues: Extrapolation Errors}

\bi
\item <1->Solving model with ``serious'' risk requires evaluating (marginal) value outside convex hull of gridpoints

\item <2->If solution method requires value and marginal value to be accurate, deviations in their extrapolation can cause compounding errors

\item <3->Always need to compute limiting solution: what do policy and (marginal) value functions asymptote to as state goes to $\infty$?

\item <3->Carroll: Solution asymptotes from below to PF problem

\item <4->Argument is complicated by non-concave value

\item <4->Does limiting solution exist with discrete choice?

\ei
\end{frame}


\begin{frame}
\frametitle{The Big Issues: Dimensionality Mismatch}

\bi
\item <1->What if end-of-period state is 2D, but decision-time is 1D?

\item <1->Not every end-of-period state is consistent with FOCs

\item <2->Requires computing expectations at a non-optimal candidate-- what EGM tries to avoid?

\item <2->Examples: Portfolio choice, NEGM, choosing house size

\item <3->Sometimes requires using value function itself

\item <3->Maybe use implicit function theorem to get a ``good guess''?
\ei
\end{frame}


\begin{frame}
\frametitle{Major Modeling Areas}

\begin{enumerate}
\item Human capital acquisition / education
	
\item Endogenous labor supply (and demand?)

\item Housing and durable goods

\item Health and insurance
\end{enumerate}

\end{frame}


\begin{frame}
\frametitle{Human Capital and Education (1/2)}

\bi
\item <1->Model: Add discrete choice about continuing education early in lifecycle, off-the-shelf consumption-saving for most of life

\item <2->Education decision depends on:
\begin{enumerate}
\item <2->Increase in productivity/wage with education

\item <2->OOP cost of education: loans / parents?

\item <2->Time and risk preferences
\end{enumerate}

\item <2->Can have ex ante heterogeneity in all those dimensions

\item <3->Education misallocation: ``dumb rich kid'' goes to college

\item <3->How would increase in education loan access or stronger quality signal affect misallocation? Macro effect?

\item <4->Limits difficulties from discrete choice

\ei
\end{frame}


\begin{frame}
\frametitle{Human Capital and Education (2/2)}

\bi
\item <1->Harder: Sector-specific human capital

\item <1->Workers can invest in education to increase sector-specific productivity

\item <1->Working in sector also increases productivity

\item <2->What happens to workers if a sector experiences an exogenous shock?

\item <2->Re-education is costly, lifecycle considerations

\item <3->Seneviratne (2013): Discrete choice, discrete state

\item <3->HARK version: Convexify state space, but not choice space
\ei
\end{frame}



\begin{frame}
\frametitle{Labor Supply Models (1/5)}

\begin{enumerate}
\item Intensive margin (working version by T.\ Magne)

\item Extensive margin (versions by MNW and P. Mogensen)

\item Job search intensity

\item Incorporating into aggregate framework

\end{enumerate}
\end{frame}


\begin{frame}
\frametitle{Labor Supply Models (2/5)}
Model of labor supply on intensive margin:
\begin{equation*}
u(c,\ell) = ((1-\ell)^\alpha c)^{1-\rho}/(1-\rho),
\end{equation*}
\begin{equation*}
v_t(b_t,\theta_t) = \max_{c_t, \ell_t} u(c_t,\ell_t) + \beta \Live_t \E_t \left[(\psi_{t+1} \Gamma_t)^{1-\rho} v_{t+1}(b_{t+1},\theta_{t+1}) \right] \text{ s.t. }
\end{equation*}
\begin{equation*}
y_t = \ell_t \theta_t, \qquad \ell_t \in [0,1],
\end{equation*}
\begin{equation*}
a_t = m_t + y_t - c_t, \hspace{0.5cm} a_t \geq \underline{a},
\end{equation*}
\begin{equation*}
b_{t+1} = R/(\Gamma_t \psi_{t+1}) a_t, 
\end{equation*}
\begin{equation*}
\psi_{t+1} \sim F_{\psi t+1}(\psi), \hspace{0.5cm} \theta_{t+1} \sim F_{\theta t+1}(\theta), \hspace{0.25cm} \E[\psi_{t+1}] = 1.
\end{equation*}
\end{frame}


\begin{frame}
\frametitle{Labor Supply Models (3/5)}
Model of labor supply on extensive margin:
\begin{equation*}
u(c,\ell) = c^{1-\rho}/(1-\rho) - \alpha \ell,
\end{equation*}
\begin{equation*}
v_t(b_t,\theta_t,\ell_{t-1}) = \max_{c_t, \ell_t} u(c_t,\ell_t) + \beta \Live_t \E_t \left[(\psi_{t+1} \Gamma_t)^{1-\rho} v_{t+1}(b_{t+1},\theta_{t+1},\ell_t) \right] \text{ s.t. }
\end{equation*}
\begin{equation*}
y_t = \ell_t \theta_t, \qquad \ell_t \in \{0,\ell_{t-1}\},
\end{equation*}
\begin{equation*}
a_t = m_t + y_t - c_t, \hspace{0.5cm} a_t \geq \underline{a},
\end{equation*}
\begin{equation*}
b_{t+1} = R/(\Gamma_t \psi_{t+1}) a_t, 
\end{equation*}
\begin{equation*}
\psi_{t+1} \sim F_{\psi t+1}(\psi), \hspace{0.5cm} \theta_{t+1} \sim F_{\theta t+1}(\theta), \hspace{0.25cm} \E[\psi_{t+1}] = 1.
\end{equation*}
\end{frame}



\begin{frame}
\frametitle{Labor Supply Models (4/5)}
Model of endogenous employment search:
\begin{equation*}
u(c,s) = ((1-s)^\alpha c)^{1-\rho}/(1-\rho),
\end{equation*}
\begin{equation*}
v_t(m_t,e_t) = \max_{c_t,s_t} u(c_t,s_t) + \beta \Live_t \E_t \left[(\psi_{t+1} \Gamma_t)^{1-\rho} v_{t+1}(m_{t+1},e_{t+1}) \right] \text{ s.t. }
\end{equation*}
\begin{equation*}
a_t = m_t - c_t, \hspace{0.5cm} a_t \geq \underline{a}, \hspace{0.5cm} s_t \in [0,1],
\end{equation*}
\begin{equation*}
m_{t+1} = R/(\Gamma^e_t \psi_{t+1}) a_t + \theta_t e_{t+1} + \underline{b}(1-e_{t+1}), 
\end{equation*}
\begin{equation*}
\Prob(e_{t+1} = 1 | e_t = 0) = s_t, \qquad \Prob(e_{t+1} = 0 | e_t = 1) = \mho,
\end{equation*}
\begin{equation*}
\psi_{t+1} \sim F^e_{\psi t+1}(\psi), \hspace{0.5cm} \theta_{t+1} \sim F_{\theta t+1}(\theta), \hspace{0.25cm} \E[\psi_t] = 1.
\end{equation*}
\end{frame}



\begin{frame}
\frametitle{Labor Supply Models (5/5)}
Applications of \texttt{Market} for labor models:
\bi
\item Non-trivial calculation of $L_t = \int_0^1 \ell_{it} P_{it} \theta_{it} di$ for Cobb-Douglas

\item Disutility of employment search and probability of job loss depend on labor market slackness

\item Can look at behavior in response to change in Social Security
\ei
\end{frame}



\begin{frame}
\frametitle{Durable Goods: Housing and Health (1/4)}

\bi
\item <1-> Durable goods produce a flow of benefits over time

\item <1-> Maybe utility, maybe something else

\item <1-> Might act as an asset if disinvestment is possible

\item <2-> Changing level of $d_t$ might involve fixed cost

\item <2-> Generates ``region of inaction'' / inactivity constraint

\item <3-> Grossman (1972): Health is a capital stock, can't disinvest

\item <3-> Health produces longevity, reduces disutility of work

\ei
\end{frame}



\begin{frame}
\frametitle{Durable Goods: Housing and Health (2/4)}

General (housing) durable goods model:
\begin{equation*}
u(c,d) = (c^\alpha,d^{1-\alpha})^{1-\rho}/(1-\rho).
\end{equation*}
\begin{equation*}
v_t(m_t,d_t) = \max_{c_t,i_t} u(c_t,d_t) + \beta \Live_t \E_t \left[(\psi_{t+1} \Gamma_t)^{1-\rho} v_{t+1}(m_{t+1},d_{t+1}) \right] \text{ s.t. }
\end{equation*}
\begin{equation*}
a_t = m_t - c_t - i_t, \hspace{0.5cm} a_t \geq \underline{a},
\end{equation*}
\begin{equation*}
D_t = d_t + g(i_t), \hspace{0.5cm} d_{t+1} = (1-\delta_{t+1})D_t, \hspace{0.5cm} \delta_{t+1} \sim F_{\delta}(\delta),
\end{equation*}
\begin{equation*}
m_{t+1} = R/(\Gamma_t \psi_{t+1}) a_t + \theta_{t+1}, 
\end{equation*}
\begin{equation*}
\psi_{t+1} \sim F_{\psi t+1}(\psi), \hspace{0.5cm} \theta_{t+1} \sim F_{\theta t+1}(\theta), \hspace{0.25cm} \E[\psi_t] = 1.
\end{equation*}
\end{frame}



\begin{frame}
\frametitle{Durable Goods: Housing and Health (3/4)}
Variations of durable goods model require different solvers:
\bi
\item <1-> Easiest case: $g(i_t)$ is convex, $i_t \in \R$.  Every end-of-period state $(a_t,D_t)$ associated with \textit{some} beginning-of-period state.

\item <1->Slightly harder: $i_t \geq 0$, must handle constraint

\item <2->Somewhat harder: $g(i_t) = i_t/\pi_t$.  One locus in $(a_t,D_t)$ space is optimal; each point on optimal $(a_t,D_t)$ locus associated with locus in $(m_t,d_t)$ space.

\item <3->Even harder: $g(i_t) = \widehat{g}(i_t) + K \mathbf{1}(i_t \neq 0)$, with $\widehat{g}(0) = 0$ and $\widehat{g}(\cdot)$ convex.  Must check $i_t=0$ soln everywhere, discont.

\item <4->Just ugh: $g(i_t) =  i_t/\pi_t +K \mathbf{1}(i_t \neq 0)$, $i_t \geq 0$.
\ei
\end{frame}


\begin{frame}
\frametitle{Durable Goods: Housing and Health (4/4)}
Applications for \texttt{Market} with housing durable goods:
\bi
\item Endogenous pricing of durable good: housing market

\item Dynamics of demand for durables after an aggregate shock
\ei

Next steps for health models:
\bi
\item White (2018) does health investment model with exogenous income (retired people only)

\item Can add income process, interact with labor supply decision

\item Health insurance and endogenous health status
\ei
\end{frame}


\begin{frame}
\frametitle{How to Begin}
\bi
\item <1->Find an appropriate adviser for your interests

\item <2->Discuss what they think the ``next big questions'' are

\item <2->Find overlap with what you want to do

\item <3->Figure out whether a structural model is necessary

\item <4->Check out HARK's tutorials and documentation

\item <4->Talk to me if you need help with HARK

\item <5->Take my Topics in Dynamic Modeling course at UDel!
\ei
\end{frame}



\end{document}